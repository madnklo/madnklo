\documentclass[11pt,a4paper]{article}
\usepackage[utf8]{inputenc}
\usepackage[T1]{fontenc}

\usepackage[margin=2cm]{geometry}
\usepackage{hyperref}
\usepackage{amssymb,mathtools}
\usepackage[extdef]{delimset}
\usepackage{physics}
\usepackage{cleveref}

% Number of spacetime dimensions
\let\olddim\dim
\renewcommand{\dim}[0]{d}
% Dimensional regularisation parameter
\newcommand{\eps}[0]{\epsilon}
% Colour operators
\newcommand{\colorT}[1]{\mathbf{T}_{#1}}
% Merged particles in intermediate mappings
\newcommand{\mgd}[1]{\widehat{#1}}

\title{Subtraction of $e^+e^- \to u\bar{u}d\bar{d}$
with distributed soft counterterms}
\author{Simone Lionetti}
\date{\today}

\begin{document}

\maketitle

\section{Counterterms}

For the process $e^+e^- \to u_1\bar{u}_2d_3\bar{d}_4$
as a real correction to $e^+e^- \to jj+X$,
the elementary limits that need to be regulated are
\begin{equation}
	C_{12}, \quad C_{34}, \quad
	C_{134}, \quad C_{234}, \quad C_{123}, \quad C_{124}, \quad
	S_{12}, \quad S_{34}.
\end{equation}
At most three of these limits have a common overlap
and need to be considered simultaneously.
Due to the lack of a tree-level diagram for $e^+e^- \to gg$,
the limit of two collinear pairs $C_{12}C_{34}$ is regular.
The maximal overlaps that we need to consider are therefore
\begin{equation}
	C_{123}S_{12}C_{12}, \quad
	C_{124}S_{12}C_{12}, \quad
	C_{134}S_{34}C_{34}, \quad
	C_{234}S_{34}C_{34}.
\end{equation}
These are all of the same type,
so it will be sufficient to consider the representative $C_{123}S_{12}C_{12}$.

\subsection{C(1,2)}
We subtract the limit of a $q\bar{q}$ pair going collinear using the current
\begin{equation}
	C_{12}^{\mu\nu} = \frac{T_R}{s_{12}} \brk[s]*{
		- g^{\mu\nu}
		+ 4 z_{1,2} z_{2,1} \frac{k_{1,2}^\mu k_{1,2}^\nu}{k_{1,2}^2}
	},
\end{equation}
in combination with the generalised rescaling mapping
and ? variables.

\subsection{C(1,2,3)}

The triple-collinear counterterm for $q'\bar{q}'q$
is determined using the current \cite{Catani:1999ss}
\begin{equation}
	C_{123} = \frac{C_F T_R}{2 s_{123}^2} \brk[s]*{
		- \frac{t_{12,3}^2}{s_{12}^2}
		+ \frac{s_{123}}{s_{12}} \brk*{
			\frac{4 z_{3,12} + (z_{1,23} - z_{2,13})^2}{z_{12,3}}
			+(1 - 2\eps) z_{12,3}
		}
		- (1 - 2\eps)
	},
\end{equation}
where
\begin{equation}
	t_{12,3}
	\equiv 2 \frac{z_{1,23} s_{23} - z_{2,13} s_{13}}{z_{12,3}}
	+ \frac{z_{1,23} - z_{2,13}}{z_{12,3}} s_{12}.
\end{equation}
Once more we use ? variables and the generalised rescaling mapping.


\subsection{S(1,2)}

In order to construct the $q\bar{q}$ soft counterterm,
we start from the form of the current used in \cite{Somogyi:2005xz} which reads
\begin{equation}
	\frac{T_R}{s_{12}^2} \sum_i \sum_j
	\frac{s_{1i}s_{2j} + s_{1j}s_{2i} - s_{12}s_{ij}}{s_{(12)i} s_{(12)j}}
	\colorT{i}\cdot\colorT{j},
\end{equation}
where the sum runs over all coloured partons of the reduced process
and includes the case $i=j$.

Before discussing partial fractioning, we observe
that the global factor $s_{12}^{-2}$ may cause the counterterm to diverge
in limits where neither $s_{12i}$ nor $s_{12j}$ go to zero.
More concretely, the contribution from the terms with $i=j$ reads
\begin{equation}
\label{eq:qqsoftieqj}
	\frac{T_R}{s_{12}^2} \sum_{i}
	\frac{2s_{1i}s_{2i}}{s_{(12)i}^2}
	\colorT{i}^2,
\end{equation}
and \emph{all} terms are divergent in the triple-collinear limit $12j$
for \emph{any} $j$.
Thus, although from \cref{eq:qqsoftieqj} one might be tempted
to assign the $i$-th term to $12i$-collinear kinematics,
every term needs to be distributed among all $12j$-collinear kinematics.
To this end, we use colour conservation to move all terms off the colour diagonal
\begin{equation}
	\sum_{i}
	\frac{2s_{1i}s_{2i}}{s_{(12)i}^2}
	\colorT{i}^2
	= - \sum_i \sum_{j\ne i}
	\frac{2s_{1i}s_{2i}}{s_{(12)i}^2}
	\colorT{i}\cdot\colorT{j}
	= - \sum_i \sum_{j\ne i} \brk[s]*{
	\frac{s_{1i}s_{2i}}{s_{(12)i}^2}
	+ \frac{s_{1j}s_{2j}}{s_{(12)j}^2}
	}
	\colorT{i}\cdot\colorT{j}.
\end{equation}
In this sense, the kinematics that we assign do not follow the divergent structure
of the invariant poles but rather the colour,
in a similar way as proposed for geometric subtraction \cite{Herzog:2018ggi}.

The complete off-diagonal soft current reads
\begin{equation}
\label{eq:qqsoftoffdiag}
	\frac{T_R}{s_{12}^2} \sum_i \sum_{j \ne i}
	\brk[s]*{
		\frac{s_{1i}s_{2j} + s_{1j}s_{2i} - s_{12}s_{ij}}{s_{(12)i} s_{(12)j}}
		- \frac{s_{1i}s_{2i}}{s_{(12)i}^2} - \frac{s_{1j}s_{2j}}{s_{(12)j}^2}
	} \colorT{i}\cdot\colorT{j}.
\end{equation}
At this stage, we observe that we may replace
the invariants $s_{(12)i} = 2p_{12}\cdot p_i$ and $s_{(12)j} = 2p_{12}\cdot p_j$
in the denominator with $s_{12i}$ and $s_{12j}$ at our leisure.
Indeed, this operation modifies the counterterm by terms which are of higher order
in the double-soft limit $S_{12}$, and therefore does not spoil the cancellation
in the counterterm's defining limit.
Using the triple invariant $s_{12i}$ seems convenient
because it makes denominators naturally match the ones of the collinear counterterm
(which cannot be changed because the modification $s_{123}\to s_{(12)3}$
is \emph{not} higher-order in the triple-collinear limit).
We thus use
\begin{equation}
\label{eq:qqsoftoffdiagmod}
	S_{12} = \frac{T_R}{s_{12}^2} \sum_{j \ne i}
	\brk[s]*{
		\frac{s_{1i}s_{2j} + s_{1j}s_{2i} - s_{12}s_{ij}}{s_{12i} s_{(12)j}}
		- \frac{s_{1i}s_{2i}}{s_{12i}^2} - \frac{s_{1j}s_{2j}}{s_{(12)j}^2}
	} \colorT{i}\cdot\colorT{j}.
\end{equation}
Whether the choice of using $s_{(12)j}$ instead of $s_{12j}$
is important elsewhere in the subtraction was not documented
and needs investigation.

It is easy to partial-fraction \cref{eq:qqsoftoffdiagmod}
into collinear kinematics.
In the present implementation we use
\begin{equation}
	1 = \frac{s_{12i}}{s_{12i} + s_{12j}} + \frac{s_{12j}}{s_{12i} + s_{12j}},
\end{equation}
for each term in the dipole sum which leads to
\begin{equation}
\label{eq:qqsoftoffdiagmodpf}
	S_{12}^{(i)} = \frac{T_R}{s_{12}^2} \sum_{j \ne i}
	\frac{s_{12j}}{s_{12i} + s_{12j}}
	\brk[s]*{
		\frac{s_{1i}s_{2j} + s_{1j}s_{2i} - s_{12}s_{ij}}{s_{12i} s_{(12)j}}
		- \frac{s_{1i}s_{2i}}{s_{12i}^2} - \frac{s_{1j}s_{2j}}{s_{(12)j}^2}
	} \colorT{i}\cdot\colorT{j}.
\end{equation}
Possibly in the future we may want to change the partial fractions
to be dependent only on angles and not on energies:
in the case of two collinear pairs with distributed single-soft limits,
this has been noted to be essential for disjoint collinear limits to work
in combination with distributed soft subtraction.


\subsection{C(S(1,2),3)}

Shifting out of the diagonal the sum over colour dipoles for the $q\bar{q}$ soft limit
turns out to be extremely practical also to take its $C_{123}$ triple-collinear limit.
We start with either \cref{eq:qqsoftoffdiagmod} or \cref{eq:qqsoftoffdiagmodpf}
(the partial fraction makes no difference in the $C_{123}$ limit), and observe that
for a given term to contribute one of $i$ or $j$ needs to be equal to $3$,
and in the collinear limit the ratio of scalar products with another leg
is equal to a ratio of momentum fractions.
After this replacement, colour conservation can be used and we find
\begin{equation}
	C_{123}S_{12} = - \frac{2T_R}{s_{12}^2} \colorT{3}^2
	\brk[s]*{
		\frac{s_{13} z_{2,13} + s_{23} z_{1,23} - s_{12} z_{3,12}}
		{s_{123} z_{12,3}}
		- \frac{s_{13}s_{23}}{s_{123}^2} - \frac{z_{1,23} z_{2,13}}{z_{12,3}^2}
	}.
\end{equation}
Note that, if the contributions on the diagonal have not been reshuffled,
some effort is needed to see that the latter two terms are needed.
We also note that subtracting this sub-limit from $C_{123}$
many terms simplify and we are left with
\begin{equation}
	C_{123} - C_{123}S_{12} = \frac{C_F T_R}{s_{123}^2} \brk[s]*{
		\frac{s_{123}}{s_{12}} \frac{z_{1,23}^2 + z_{2,13}^2}{z_{12,3}} - 1
		+ \eps \brk*{1 + \frac{s_{123}}{s_{12}} z_{12,3} }
	},
\end{equation}
which is what is currently implemented in the code (for $\eps=0$).
This hard triple-collinear counterterm is clearly associated
to $123$-collinear kinematics, and for the simplifications to occur
the momentum fractions have to be computed as in $C_{123}$.


\subsection{C(C(1,2),3)}

The strong-ordered collinear limit $C(C(1,2),3)$ is the first nested limit
that we encounter whose counterterm we implement in an iterated fashion.
To this end we follow the steps of \cite{Somogyi:2006da}.
Starting from the counterterm $C(1,2)$, we take the extra collinear limit
of the parent gluon $\mgd{12}$ of the quark-antiquark pair
going collinear to the mapped, different-species quark $\mgd{3}$.
This involves taking the collinear limit of a spin-correlated matrix element,
which gives the splitting function
\begin{equation}
	C_{\mgd{12}\mgd{3}}^{\alpha\beta,ss'} = \frac{C_F}{s_{\mgd{12}\mgd{3}}}
	\delta_{ss'} \brk[s]*{
		\frac{z_{\mgd{12},\mgd{3}}}{2} d^{\alpha\beta}
		- 2 \frac{z_{\mgd{3},\mgd{12}}}{z_{\mgd{12},\mgd{3}}}
		\frac{k_{\mgd{3},\mgd{12}}^\alpha k_{\mgd{3},\mgd{12}}^\beta}
		{k_{\mgd{3},\mgd{12}}^2}
	}.
\end{equation}
The sum over physical polarisations is given by the transverse tensor
with respect to the light-cone vector in the collinear direction $p$
and a reference null vector $n$,
\begin{equation}
	d^{\alpha\beta}(p, n) \equiv
	-g^{\alpha\beta}
	+ \frac{p^\alpha n^\beta + p^\beta n^\alpha}{p \cdot n}.
\end{equation}
We have indicated with a hat the variables which are computed
after merging particles 1 and 2.
Performing the Lorentz algebra we find
\begin{equation}
	C_{\mgd{12}\mgd{3}}^{\alpha\beta,ss'} C_{12,\alpha\beta}
	= \frac{C_FT_R}{s_{12}s_{\mgd{12}\mgd{3}}} \brk[s]3{
		\brk3{
			2 \frac{z_{\mgd{3},\mgd{12}}}{z_{\mgd{12},\mgd{3}}}+
			z_{\mgd{12},\mgd{3}} (1 - \eps)
		} - 2 z_{1,2} z_{2,1} \brk3{
			z_{\mgd{12},\mgd{3}} +
			\frac{z_{\mgd{3},\mgd{12}}}{z_{\mgd{12},\mgd{3}}}
			\frac{(2k_{1,2}\cdot k_{\mgd{3},\mgd{12}})^2}
			{k_{1,2}^2 k_{\mgd{3},\mgd{12}}^2}
		}
	}.
\end{equation}
The reduced matrix element is evaluated for momenta that have been obtained
merging particles 1, 2 and 3 with a generalised rescaling mapping.
This is equivalent to merging 1 and 2 into $\mgd{12}$
recoiling against all other legs,
and later merging $\mgd{12}$ with $\mgd{3}$
recoiling against all remaining momenta.


\subsection{S(C(1,2))}

The limit where the $q\bar{q}$ pair is both collinear and soft is over-subtracted
and needs to be added back.
The corresponding counter-counterterm $S(C(1,2))$
may also be constructed iteratively as done in \cite{Somogyi:2005xz}.
After the $C_{12}$ limit has been taken the reduced, spin-correlated matrix element 
which contains the single parent $\mgd{12}$ in the limit of soft $\mgd{12}$
factorises with the current
\begin{equation}
	S_{\mgd{12}}^{\mu\nu}
	= \sum_{i, j}
	\frac{\mgd{p}_i^\mu \mgd{p}_j^\nu + \mgd{p}_i^\nu \mgd{p}_j^\mu}
	{s_{\mgd{12}\mgd{i}} s_{\mgd{12}\mgd{j}}}
	2\colorT{i}\cdot\colorT{j}.
\end{equation}

Since this current multiplies the splitting function for the 12-collinear limit
which features a factor $s_{12}^{-1}$,
the counterterm is divergent in all triple-collinear configurations
and not just in the $12i$- or $12j$-collinear limits.
Similarly to the case of $S_{12}$, it is thus convenient
to shift away the elements on the colour diagonal using colour conservation,
which gives
\begin{equation}
	S_{\mgd{12}}^{\mu\nu}
	= \sum_{j \ne k} \brk[s]*{
		\frac{\mgd{p}_i^\mu \mgd{p}_j^\nu + \mgd{p}_i^\nu \mgd{p}_j^\mu}
		{s_{\mgd{12}\mgd{i}} s_{\mgd{12}\mgd{j}}}
		- \frac{\mgd{p}_i^\mu \mgd{p}_i^\nu}{s_{\mgd{12}\mgd{i}}^2}
		- \frac{\mgd{p}_j^\mu \mgd{p}_j^\nu}{s_{\mgd{12}\mgd{j}}^2}
	} 2 \colorT{i}\cdot\colorT{j}.
\end{equation}
\begin{equation}
	S_{12} C_{12} = \frac{T_R}{s_{12}} \sum_{j \ne k} \brk[s]*{
		\frac{s_{jk}}{s_{12j}s_{12k}}
		-z(1-z)\brk*{
		\frac{2 s_{j\perp} s_{k\perp}}{s_{12j} s_{12k} n_\perp^2}
		- \frac{s_{j\perp}^2}{s_{12j}^2 n_\perp^2}
		- \frac{s_{k\perp}^2}{s_{12k}^2 n_\perp^2}
		}
	} \colorT{j}\cdot\colorT{k},
\end{equation}


\subsection{C(S(C(1,2)),3)}


\begin{thebibliography}{9}

%\cite{Catani:1999ss}
\bibitem{Catani:1999ss} 
  S.~Catani and M.~Grazzini,
  %``Infrared factorization of tree level QCD amplitudes at the next-to-next-to-leading order and beyond,''
  Nucl.\ Phys.\ B {\bf 570}, 287 (2000)
  doi:10.1016/S0550-3213(99)00778-6
  [hep-ph/9908523].
  %%CITATION = doi:10.1016/S0550-3213(99)00778-6;%%
  %227 citations counted in INSPIRE as of 04 Apr 2019

%\cite{Somogyi:2005xz}
\bibitem{Somogyi:2005xz} 
  G.~Somogyi, Z.~Trocsanyi and V.~Del Duca,
  %``Matching of singly- and doubly-unresolved limits of tree-level QCD squared matrix elements,''
  JHEP {\bf 0506}, 024 (2005)
  doi:10.1088/1126-6708/2005/06/024
  [hep-ph/0502226].
  %%CITATION = doi:10.1088/1126-6708/2005/06/024;%%
  %111 citations counted in INSPIRE as of 04 Apr 2019

%\cite{Somogyi:2006da}
\bibitem{Somogyi:2006da} 
  G.~Somogyi, Z.~Trocsanyi and V.~Del Duca,
  %``A Subtraction scheme for computing QCD jet cross sections at NNLO: Regularization of doubly-real emissions,''
  JHEP {\bf 0701}, 070 (2007)
  doi:10.1088/1126-6708/2007/01/070
  [hep-ph/0609042].
  %%CITATION = doi:10.1088/1126-6708/2007/01/070;%%
  %115 citations counted in INSPIRE as of 04 Apr 2019
  
\bibitem{Herzog:2018ggi}
	F.~Herzog,
	Geometric subtraction for real radiation at NNLO,
	\url{https://www.ggi.infn.it/talkfiles/slides/slides4304.pdf}
 
\end{thebibliography}

\end{document}
